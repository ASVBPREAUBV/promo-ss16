\documentclass{article}
% \usepackage{ngerman}
\usepackage[utf8]{inputenc}
\usepackage{enumerate}
\usepackage{courier}
% \usepackage{paralist}
% \usepackage{amsmath}
% \usepackage{amsfonts}
\usepackage{minted}

\title{Übungen zu Rekursion - Programmierung und Modellierung 2016}

\begin{document}

\maketitle

\begin{center}
\textit{Besprechung am 22.Juli 2016}
\end{center}

\section*{Aufgabe 1}
Definieren sie folgende Funktionen:
\begin{enumerate} [a)]
    \item \begin{minted}{haskell}
intersperse :: a -> [a] -> [a]
          \end{minted}
    Diese Funktion nimmt ein Separator, eine Liste und packt zwischen jedes Element den Seperator. Es kommt kein Separator vor das erste oder hinter das letzte Element der Liste.
    
    \begin{minted}{haskell}
        > intersperse ',' "hallo"
        "h,a,l,l,o"
        
        > intersperse 0 [1,2,3]
        [1,0,2,0,3]
        
        > intersperse 0 [1]
        [1]
    \end{minted}

    \item \begin{minted}{haskell}
at :: [a] -> Int -> a
          \end{minted}
    Diese Funktion gibt mir das Element am jeweiligen Index der Liste zurück. \\
    Fehlerbehandlung sind nicht nötig. \texttt{(!!)} darf nicht benutzt werden. \\

    \begin{minted}{haskell}
        > at [1,2,3] 0
        1

        > at "hallo" 4
        'o'
    \end{minted}

\newpage
    \item \begin{minted}{haskell}
take :: Int -> [a] -> [a]
          \end{minted}
    Diese Funktion nimmt eine Zahl \texttt{n} und eine Liste \texttt{xs} und gibt den Prefix der Liste mit Länge \texttt{n} zurück
    oder die Liste selbst, falls \texttt{n > length xs}. Fehlerbehandlung sind nicht nötig. \\

    \begin{minted}{haskell}
        > take 3 [1,2,3,4,5]
        [1,2,3]
       
        > take 10 [1,2,3]
        [1,2,3]
       
        > take 0 [1,2,3]
        []
    \end{minted}

    \item \begin{minted}{haskell}
repeat :: a -> [a]
          \end{minted}
    Diese Funktion nimmt ein Argument \texttt{x} und erstellt eine unendlich große Liste mit aus \texttt{x}'n
    \begin{minted}{haskell}
        -- Pseudobeispiel:
        --   repeat 1 => [1,1,1,1,1...

        > take 10 (repeat 1)
        [1,1,1,1,1,1,1,1,1,1]

        > take 5 (repeat 'a')
        "aaaaa"

        > take 2 (repeat [])
        [[],[]]
    \end{minted}

    \item \begin{minted}{haskell}
divBy3 :: [Int] -> [Int]
          \end{minted}
    Definieren sie mithilfe von List-Comprehentions eine Funktion die alle durch 3 teilbaren Zahlen zurückgibt.
    \begin{minted}{haskell}
        > divBy3 [1..10]
        [3,6,9]
       
        > divBy3 [(-1),(-2)..(-10)]
        [-3,-6,-9]
    \end{minted}

\end{enumerate}

\end{document}