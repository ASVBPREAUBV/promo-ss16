\documentclass{article}
% \usepackage{ngerman}
\usepackage[utf8]{inputenc}
\usepackage{enumerate}
\usepackage{courier}
% \usepackage{paralist}
% \usepackage{amsmath}
% \usepackage{amsfonts}
\usepackage{minted}
\usepackage{mdframed}

\definecolor{bg}{rgb}{0.85,0.95,0.85}

\title{Altklausur Bry 2012 Remastered für Programmierung und Modellierung 2016}
\author{Alexander Isenko}
\begin{document}

\maketitle

\begin{center}
\textit{Besprechung am 22.Juli 2016}
\end{center}

\section*{Aufgabe 1}
Hier wird nach Funktionen gefragt die ihr euch selber einfallen lassen könnt, total egal wie kompliziert
oder einfach sie sind, sie müssen lediglich den Bedingungen entsprechen.
\begin{enumerate} [a)]
    \item  Definieren sie eine monomorphe Funktion (keine Typvariablen, nur feste Typen)
    \item Definieren sie eine polymorphe Funktion (mit Typvariablen)
        \begin{enumerate}[i)]
            \item parametrisch polymorph  (keine Typvariablen vor ($=>$)) \\
            \item ad-hoc polymorph       (mid. eine Typvariable vor ($=>$)) \\
        \end{enumerate}
    \item  Definieren sie eine \textit{gecurriete} Funktion (eine Funktion die ein Tupel nimmt, aber mit \texttt{curry} stattdessen die Argumente hintereinander akzeptiert)
        \begin{minted}[linenos]{haskell}
-- Hilfestellung:

   > :t curry
   curry :: ((a, b) -> c) -> a -> b -> c
        \end{minted}
\end{enumerate}

\newpage

\section*{Aufgabe 3}
Definieren sie die Funktion \texttt{reverse} für Listen in Haskell.
    \begin{minted}[linenos]{haskell}
        > :t reverse
        [a] -> [a]

        > reverse "hallo"
        "ollah"
        
        > reverse [1,2,3]
        [3,2,1]
        
        > reverse []
        []
    \end{minted}

\section*{Aufgabe 4}
Sei folgender Code gegeben, was ist das Ergebnis von \texttt{res}?
\begin{minted}[linenos]{haskell}
y = 5
x = 2
goo y     = x * y
fuu (x,y) = x + goo y
res = (goo y, fuu (5,7))
\end{minted}

\section*{Aufgabe 6}
Definieren sie das Standartskalarprodukt mithilfe von \texttt{zip} und \texttt{Data.List.foldl/foldr}

\begin{minted}[linenos]{haskell}
--  Typsignatur:
--
skalarProdukt :: Num a => [a] -> [a] -> a
--
--  Beispiel:
--
skalarProdukt [1,2,3] [4,5,6] = 1*4 + 2*5 + 3*6 = 4 + 10 + 18 = 32

-- Hilfestellung:
--
zip :: [a] -> [b] -> [(a,b)]
zip []     []     = []
zip (x:xs) (y:ys) = (x,y) : zip xs ys




foldl :: (b -> a -> b) -> b -> [a] -> b
foldl f acc []     = acc
foldl f acc (x:xs) = foldl f (f acc x) xs

foldr :: (a -> b -> b) -> b -> [a] -> b
foldr f acc []     = acc
foldr f acc (x:xs) = f x (foldr f acc xs)
\end{minted}

\section*{Aufgabe 7}
Sei folgender Code gegeben.
\begin{minted}[linenos, escapeinside=||]{haskell}
ks = [3,5,7]

pred = ||\||x -> x < 0

f p x (y:ys) = if p y
                   then f (not . p) x ys
                   else f p         x ys
f p x _      = p x

g x = let f x = x
      in g (f x)
\end{minted}

Geben sie wenn möglich den Wert bzw. den Typ der folgenden Ausdrücke an:

\begin{enumerate} [a)]
    \item \texttt{f} \\[2mm]
    \item \texttt{f pred 0 []} \\[2mm]
    \item \texttt{f pred 0 ks} \\[2mm]
    \item \texttt{g} \\[2mm]
    \item \texttt{g ks} \\[2mm]
\end{enumerate}
\end{document}